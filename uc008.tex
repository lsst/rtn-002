\subsection{S-008: Correcting Misconceptions Regarding Functionality}

\textbf{Issue origin and description} \\
A team of CST members leads an RSP/Rubin data training workshop, and becomes aware of a common misconception or misuse regarding RSP functionality via interaction with the attendees.

\textbf{Success scenario(s)} \\
The relevant documentation and tutorials are updated and this issue is not encountered in the future.

\textbf{Success workflow)} \\
Step 1. A CST member opens an Issue Ticket and describes the scenario, including contact information for attendees. \\
Step 2. Work proceeds by CST members to generate more appropriate documentation and tutorials that demonstrate how to use the relevant functionality. \\
Step 3. The new materials are released and are circulated back to the original attendees.

\textbf{Alternative success scenario} \\
Alternative step 2: For the cases where the opportunity to update the functionality itself, to make its use more intutitive, is possible, CST members will propose this to Rubin developers and then also update relevant documentation and tutorials.

\textbf{Failure scenario(s)} \\
Users perpetuate the misconception or misues of the RSP functionality.

\textbf{Risks} \\
Persistent issues will reduce efficient RSP use.

\textbf{Preconditions for success} \\
CST members must be in attendance at relevant workshops. \\
The CST understands RSP functionality well enough to spot misuse.