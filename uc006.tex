\textbf{Use Case:} 006 Survey Strategy Alteration for Photometric Redshifts \\

\textbf{Characteristic Information} \\
\textbf{Trigger:} DESC photo-z and weak lensing working groups determine that their projected photo-z statistics fall short of the LSST SRD year 10 target values for photo-z quality, and the DESC spokesperson communicates this formally to the SP-CST lead via email. \\
\textbf{Goal in Context:} As the Rubin Observatory intends to meet the science requirements of its LSST survey, staff will need to study and implement changes to the survey strategy to acquire the necessary data to meet the photometric redshift statistics requirements. \\
\textbf{Primary Actor:} SP-CST lead, upon receipt of the email, opens an issue in an issue tracking system. \\
\textbf{Scope:} This use case defines the interaction of the survey performance team with the observatory operations team as a formal request for a change comes from the science community. \\
\textbf{Level:} Summary \\
\textbf{Preconditions:} The Observatory is in full operations, and data has been released to the science collaborations. This scenario likely takes place in year 4 of the survey, with year 3 data under study by the DESC. It assumes a typical lag of about one year for science collaborations to analyze data and provide formal feedback. Most scientists believe that limitations in photo-zs can be improved by changing the data acquisition strategy rather than algorithms. \\
\textbf{Success End Condition:} A year's worth of new data improves the photo-z statistics as expected, and the DESC photo-z and weak lensing working groups project the 10-year data to meet the requirements. \\
\textbf{Failed End Condition:} The photo-z statistics remain insufficient. The SRD photo-z requirements are not met, making the photo-z statistics the leading systematic limiting the weak lensing cosmology results. This restricts the precision of the resulting cosmological parameters. \\

\textbf{Main Success Scenario} \\
0. The DESC determines there is an issue with the photo-zs and communicates this to the SP-CST. \\
1. SP-CST creates an issue ticket assigned to the Community Scientist with expertise in cosmology, adding relevant watchers from the SP Survey Scheduling Team (SP-SST), Data Production Algorithms and Pipeline Team (DP-AP), and Observatory Operations Observatory Science Team (OO-OS). \\
2. A meeting is organized with DESC scientists, SP-CST, SP-SST, DP-AP, OO-OS representatives, and members of the Survey Cadence Optimization Committee (SCOC) to decide on next steps. It is decided to perform simulations of a cadence that increases cumulative exposure times in the u and y bands. \\
3. The SP-SST publishes a Community post describing the issue and the planned actions and responds to the DESC spokesperson. \\
4. The SP-SST generates the OpSim results with standard metrics. \\ 
5. The DESC (and other science collaborations) evaluate the simulations to determine the impact on weak lensing and photo-z issues, as well as other science cases, and submit reports back to the SCOC. \\
6. The SCOC meets (\approx 6 months) and decides to recommend a u-band focus for year 9 to the Directorate. A written summary of their discussion is provided to the SP-CST. \\
7. The SCOC written summary is made publicly available by the SP-CST. \\
8. The issue ticket is closed. \\

\textbf{Extensions} \\
\\

\textbf{Sub-Variations} \\
\\

\textbf{Related Information} (Optional) \\
\textbf{Priority:} This issue is of high priority but is not particularly urgent. \\
\textbf{Performance Target:} The time from opening to closing the ticket should be less than 1 year. \\
\textbf{Frequency:} Issues affecting the wide-field cadence are likely to arise every couple of years. \\
\textbf{Superordinate Use Case:} \\
\textbf{Subordinate Use Cases:} \\
\textbf{Channel to Primary Actor:} Interactive (email) \\
\textbf{Secondary Actors:} SP-SST, DP-AP, OO-OS, SCOC \\
\textbf{Channel to Secondary Actors:} Interactive (JIRA) \\

\textbf{Open Issues} (Optional) \\

\textbf{Schedule} \\
\textbf{Due Date:} \\
