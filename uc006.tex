{\bf Use Case:} 006 Survey Strategy Alteration for Photometric Redshifts \\

{\bf Characteristic Information} \\
Trigger: DESC photo-z and weak lensing working groups determine that their projected photo-z statistics fall short of the LSST SRD year 10 target values for photo-z quality, and the DESC spokesperson communicates this formally to the SP-CET lead via email. \\
Goal in Context: As the Rubin Observatory intends to meet the science requirements of its LSST servey, staff will need to study and implementent changes to the survey strategy to aquire the right data to meet the photometric redshift statistics requirements. \\
Primary Actor: SP-CET lead, upon receipt of email, opens an issue in an issue tracking system. \\
Scope: We are designing the interaction of the survey performance team with the observatory operations team as a formal request for a change comes from teh science community. \\
Level: Summary \\
Preconditions: This scenario envisions the Observator is in full operations and that data has been released to the science collaborations.  Likely this is year 4 of the survey and year 3 data is under study by the DESC.  It takes time for science collaborations to understand the limitations of the data and the likely lag from data release to formal feedback is a year. The scenario envisions that most scientists believe that the limitations of the photo-zs can be improved by changing the kind of data being acquired as opposed to changing the algtoithms. \\
Success End Condition: A year's worth of new data improves the photo-z statistics as expected, and the DESC photo-z and weak lensing working groups can project the 10 year data meeting requirements. \\
Failed End Condition: The photo-z statistics remain stubbornly high.  The SRD photo-z requirements are not met, and the leading systematic on the weak lensing cosmology results is the photo-z statistics.  This limits the precision of the resulting cosmological parameters. \\

{\bf Main Success Scenario} \\
0. The DESC determines there is an issue in the photo-zs and communicates this to the SP-CET. \\
1. SP-CET creates an issue ticket assigned to the Community Scientist with expertise in cosmology adding relevant watchers from the SP Survey Scheduling Team (SP-SST), Data Production Algorithms and Pipeline Team (DP-AP), and Observatory Operations Observatory Science Team (OO-OS) \\
2. A meeting is organized with DESC scientists, SP-CET, SP-SST, DP-AP, OO-OS representatives, and members of the Survey Cadence Optimization Commitee (SCOC) with the agenda of deciding what the next steps are. It is decided to perform simulations of a cadence that increases u and y band cumulative exposure times. \\
3. The SP-SST publishes a Community post describing the issue and the actions being taken and responds to the DESC spokesperson. \\
4. The SP-SST generates the OpSim results with standard metrics. \\ 
5. The DESC (and other science collaborations) evaluate the simulations to determine impact on the issues with the weak lensing & photo-z as well as on all other science cases of concern and sumbits reports back to teh SCOC. \\
6. The SCOC meets (~6months) and decides to recommend a u-band focus for year 9 to the Directorate. The written summary of their discussion is provided to the SP-CET. \\
7. The SCOC written summary is made publicly available by the SP-CET. \\
8. The issue ticket is closed. \\


{\bf Extensions} \\
\\

{\bf Sub-Variations} \\
\\

{\bf Related Information} (Optional) \\
Priority: This issue is of high priority but is not particularly urgent. \\
Performance Target: The time from opening to closing the ticket should be <1 year. \\
Frequency: Issues affecting the wide field cadence are likely every couple of years. \\
Superordinate Use Case:  \\
Subordinate Use Cases: \\
Channel to primary actor: interactive (email) \\
Secondary Actors: SP-SST, DP-AP, OO-OS, SCOC \\
Channel to Secondary Actors: interactive (JIRA) \\

{\bf Open Issues} (Optional) \\

{\bf Schedule} \\
Due Date: \\
