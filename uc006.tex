\subsection{S-006: Survey Strategy Alteration for Photometric Redshifts}

\textbf{Issue origin and description} \\
The DESC photo-z and weak lensing working groups determine that their projected photometric redshift (photo-z) statistics fall short of the LSST Science Requirements Document (SRD) year 10 target values for photo-z quality.
The DESC spokesperson communicates this issue formally to the SP-CST lead via email.

\textbf{Success scenario(s)} \\
A year's worth of new data improves the photo-z statistics as expected, enabling the DESC photo-z and weak lensing working groups to project that 10-year survey data will meet the SRD requirements.

\textbf{Success workflow} \\
Step 1. The DESC determines that there is an issue with the photo-zs and communicates it to the SP-CST. \\
Step 2. SP-CST creates an issue ticket and assigns it to the Community Scientist with expertise in cosmology, adding relevant watchers from the SP Survey Scheduling Team (SP-SST), Data Production Algorithms and Pipeline Team (DP-AP), and Observatory Operations Observatory Science Team (OO-OS). \\
Step 3. A meeting is organized with DESC scientists, SP-CST, SP-SST, DP-AP, OO-OS representatives, and members of the Survey Cadence Optimization Committee (SCOC) to decide on the next steps.
It is decided to perform simulations of a cadence that increases cumulative exposure times in the u and y bands. \\
Step 4. The SP-SST publishes a Community post describing the issue and the planned actions and responds to the DESC spokesperson. \\
Step 5. The SP-SST generates OpSim results with standard metrics. \\
Step 6. The DESC and other science collaborations evaluate the simulations to determine the impact on weak lensing and photo-z issues, as well as on other science cases, and submit reports back to the SCOC. \\
Step 7. The SCOC meets (approximately 6 months later) and decides to recommend a u-band focus for year 9 to the Directorate.
A written summary of their discussion is provided to the SP-CST. \\
Step 8. The SCOC written summary is made publicly available by the SP-CST. \\
Step 9. The issue ticket is closed. \\

\textbf{Alternative success scenario} \\
The simulations and discussions result in a different cadence modification being proposed, which still leads to the photo-z statistics improving to meet SRD requirements.

\textbf{Failure scenario(s)} \\
The photo-z statistics remain insufficient, and the SRD photo-z requirements are not met. \\
The photo-z statistics become the leading systematic limiting weak lensing cosmology results, restricting the precision of cosmological parameters. \\

\textbf{Risks} \\
Delays in addressing the issue could result in further data collection that does not contribute to improving photo-z statistics. \\
Ineffective cadence modifications could negatively impact other science goals. \\
Persistent photo-z issues could erode the community’s confidence in Rubin's ability to meet its science requirements.

\textbf{Related information (optional)} \\
\textbf{Priority:} High priority but not particularly urgent. \\
\textbf{Performance Target:} The issue should be resolved within one year of ticket creation. \\
\textbf{Frequency:} Issues affecting wide-field cadence are likely to arise every couple of years. \\

\textbf{Preconditions for success} \\
The Rubin Observatory is in full operations. \\
DESC has received sufficient data (e.g., year 3 data) to analyze and provide feedback. \\
The SP-SST, DP-AP, and SCOC are prepared to respond to cadence-related feedback from science collaborations.\\
