\subsection{S-002: Generic CST-provided help}

\textbf {Issue origin and description} \\
User posts a request for help with a simple camera issue in the Forum's Support category. \\

\textbf {Success scenario(s)} \\
A CST member posts a solution within a couple of days. \\

\textbf {Success workflow} \\
Step 1. A user makes a post in the Forum detailing their issue. \\
Step 2. A CST member responds within 24 hours to affirm the issue is being addressed. \\
Step 3. The CST member reaches out in the staff Slack space to camera team members. \\
Step 4. The CST member compiles responses from othe camera team and posts a response. \\
Step 5. The user confirms their question is answered and marks the solution. \\

\textbf {Alternative success scenario} \\
A community member solves the issue before a Rubin staff member. \\
The CST member finds the answer in camera team documentation and posts it. \\

\textbf {Failure scenario(s)} \\
The reported issue remains unresolved. \\

\textbf {Risks}\\
If the CST misses or cannot solve the issue, use of the Forum will decrease. \\
If users are unable to get support for science, LSST will be less impactful. \\

\textbf {Related information (optional)} \\
\textbf{Priority}: High \\
\textbf{Timeframe}: Days \\
\textbf{Frequency}: High \\

\textbf {Preconditions for success}\\
The Rubin Community Forum had to exist. \\
The Forum has to be used by community members. \\
The CST member had to liaise internally with camera team members (or relevant documentation had to exist). \\
