\textbf{Use Case:} 003 Observatory Personnel Detect a Fault, Leading to a Change in the Alert Production Pipeline \\

\textbf{Characteristic Information} \\
\textbf{Trigger:} A fault in one of the Raft Electronics Boards (a sensor). \\
\textbf{Goal in Context:} This fault causes an abnormally high number of DIASource detections for that sensor, most of which are appropriately flagged as artifacts by the real/bogus algorithm and do not become Alerts, but some are not flagged and are released. \\
\textbf{Primary Actor:} The System Performance Verification and Validation (SP-VV) Lead Scientist \\
\textbf{Scope:} Alert production, broker community \\
\textbf{Level:} \\
\textbf{Preconditions:} Operations \\
\textbf{Success End Condition:} The sensor anomaly is resolved by modifying the Alert Production (AP) pipeline. \\
\textbf{Failed End Condition:} Bogus DIASources from this sensor fault continue to be released as Alerts. \\

\textbf{Main Success Scenario} \\
Step 1: Rubin Observatory Operations team monitors verification and validation outputs during the night. \\
Step 2: Night staff notice an anomalous spike in the number of DIASources detected. \\
Step 3: Night staff run diagnostics to isolate the excess DIASources to a single sensor, identify the underlying fault, and determine that the hardware cannot be quickly/easily fixed (i.e., the sensor will have to be replaced). \\
Step 4: Night staff summarize their findings for the SP-VV Lead Scientist. \\
Step 5: SP-VV team runs further diagnostics and coordinates with Data Production (DP) to plan a fix. \\
Step 6: DP implements the fix (e.g., retraining the real/bogus characterization). \\
Step 7: SP-VV and DP coordinate with the CST to summarize the issue, its fix, and potential science impacts for the community. \\

\textbf{Extensions} \\
\textbf{Alteration 5.1:} IF a significant amount of the anomalous DIASources were released as Alerts THEN the SP-VV team coordinates with the CST to prepare an initial statement describing the fault and its impact on the Alert stream, to be posted to the Community Forum and sent to a community brokers email list. \\

\textbf{Sub-Variations} \\
\textbf{Variation 3.1:} IF the sensor anomaly is fixed with hardware THEN a software fix might not be necessary. \\

\textbf{Related Information} (Optional) \\
\textbf{Priority:} High \\
\textbf{Performance Target:} Identify the scope of the issue within one day; apply a fix to the AP within a week. \\
\textbf{Frequency:} Low \\
\textbf{Superordinate Use Case:} \\
\textbf{Subordinate Use Cases:} \\
\textbf{Channel to Primary Actor:} Internal discussions via Slack channels \\
\textbf{Secondary Actors:} CST, DP, science users \\
\textbf{Channel to Secondary Actors:} Posts to science users via the Community Forum \\

\textbf{Open Issues} (Optional) \\

\textbf{Schedule} \\
\textbf{Due Date:} \\
