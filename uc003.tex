\subsection{S-003: A Camera Fault Requires Modification of the Alert Production Pipeline}

\textbf {Issue origin and description} \\
A fault in one of the Raft Electronics Boards (in a sensor) causes an abnormally high number of DIASource detections for that sensor.
Most of these detections are flagged as artifacts by the real/bogus algorithm and do not become Alerts, but some are not flagged and are released. \\

\textbf {Success scenario(s)} \\
The sensor anomaly is resolved by modifying the Alert Production (AP) pipeline, ensuring that bogus DIASources are no longer released as Alerts. \\

\textbf {Success workflow} \\
Step 1. Rubin Observatory Operations team monitors verification and validation outputs during the night.\\
Step 2. Night staff notice an anomalous spike in the number of DIASources detected.\\
Step 3. Night staff run diagnostics to isolate the excess DIASources to a single sensor, identify the underlying fault, and determine that the hardware cannot be quickly or easily fixed (i.e., the sensor will need to be replaced).\\
Step 4. Night staff summarize their findings for the SP-VV Lead Scientist.\\
Step 5. The SP-VV team runs further diagnostics and coordinates with Data Production (DP) to plan a fix.\\
Step 6. DP implements the fix (e.g., retraining the real/bogus characterization).\\
Step 7. SP-VV and DP coordinate with the CST to summarize the issue, the fix, and potential science impacts for the community.\\

\textbf {Alternative success scenarios} \\
The sensor anomaly is fixed with hardware such that a software fix might not be necessary.\\
If a significant number of the anomalous DIASources were released as Alerts, the SP-VV team coordinates with the CST to prepare a public statement describing the fault and its impact on the Alert stream.
This statement is posted to the Community Forum and sent to a community brokers email list.\\

\textbf {Failure scenario(s)} \\
Bogus DIASources from the faulty sensor continue to be released as Alerts, undermining the reliability of the alert stream.\\

\textbf {Risks}\\
If the fault is not addressed, the continued release of bogus DIASources may erode trust in the alert system.\\
Additional workload for the CST and DP teams to mitigate the downstream effects of the faulty alerts.\\
Science users may waste time and resources investigating bogus alerts.\\

\textbf {Related information (optional)} \\
Priority: High \\
Timeframe: Identify the scope of the issue within one day; apply a fix to the AP within a week. \\
Frequency: Low \\

\textbf {Preconditions for success}\\
The Rubin Observatory is in operations. \\
Verification and validation outputs are actively monitored by the Operations team. \\
Communication channels (e.g., Slack, Community Forum) are established between SP-VV, CST, and DP teams. \\
