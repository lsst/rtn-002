{\bf Use Case:} 003 Observatory Personnel Detect a Fault, Leads to Change in Alert Production Pipeline \\

{\bf Characteristic Information} \\
Trigger: A fault in one of the Raft Electronics Boards (a sensor). \\
Goal in Context: This fault causes and abnormally high number of DIASource detections for that sensor, most of which are appropriately flagged as artifacts by the real/bogus algorithm and do not become Alerts, but some are not flagged and are released. \\
Primary Actor: The System Performance Verification and Validation (SP-VV) Lead Scientist \\
Scope: alert production, broker community \\
Level: \\
Preconditions: Operations \\
Success End Condition: The sensor anomaly is solved by modifying the Alert Production (AP) pipeline. \\
Failed End Condition: Bogus DIASources from this sensor fault continue to be released as Alerts. \\

{\bf Main Success Scenario} \\
Step 1: Rubin Observatory Operations team monitors verification and validation outputs during the night. \\
Step 2: Night staff notice an anomalous spike in the number of DIASources detected. \\
Step 3: Night staff run diagnostics to isolate the excess DIASources to a single sensor, identify the underlying fault, and determine that the hardware cannot be quickly/easily fixed (i.e., sensor will have to be replaced). \\
Step 4: Night staff summarize their findings for the SP-VV Lead Scientist. \\
Step 5: SP-VV team runs further diagnostics and coordinates with Data Production (DP) to plan a fix. \\
Step 6: DP implements the fix (e.g., retraining the real/bogus characterization). \\
Step 7: SP-VV and DP coordinate with the CET to summarize the issue, it's fix, and potential science impacts for the community. \\

{\bf Extensions} \\
Alteration 5.1: IF a significant amount of the anomalous DIASources were released as Alerts THEN SP-VV team coordinates with the CET to prepare an initial statement describing the fault and it's impact on the Alert stream, to be posted to the Community Forum and sent to a community brokers email list. \\

{\bf Sub-Variations} \\
Variation 3.1: IF the sensor anomaly is fixed with hardware THEN a software fix might not be necessary \\

{\bf Related Information} (Optional) \\
Priority: High  \\
Performance Target: identify scope of issue within one day; apply fix to AP within a week \\
Frequency: Low \\
Superordinate Use Case:  \\
Subordinate Use Cases: \\
Channel to primary actor: Internal discussions via Slack channels \\
Secondary Actors: CET, DP, science users \\
Channel to Secondary Actors: posts to the science users via the Community Forum \\

{\bf Open Issues} (Optional) \\

{\bf Schedule} \\
Due Date: \\
