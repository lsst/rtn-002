\subsection{S-007: A Science-Driven Modification to a Data Product}

\textbf{Issue origin and description} \\
A manuscript is published in the literature that highlights an issue with Rubin Observatory Data Release products.
This paper is brought to the attention of a CST member, who passes it along to the Community Scientist and Science Collaboration with the relevant expertise.

\textbf{Success scenario(s)} \\
The Data Release Production (DRP) team implements and tests an algorithmic change that resolves the issue, improving detection efficiency or data quality without adversely affecting other scientific data products.

\textbf{Success workflow} \\
Step 1. A manuscript is published in the literature highlighting a systematic offset in the detection efficiency of nuclear transients in different types of host galaxies. \\
Step 2. The issue is brought to the attention of a CST member, who creates an Issue Ticket assigned to the Community Scientist(s) and Science Collaboration(s) with the relevant expertise (e.g., time-domain science and the TVS SC). \\
Step 3. The Community Scientist and other relevant parties investigate the issue and identify a possible cause in the DM pipelines. \\
Step 4. The ticket is discussed with DM developers, who identify potential improvements to mitigate the issue and consider the side effects of implementing those changes. \\
Step 5. DM developers implement the code changes on a ticket branch, and the relevant CST and/or TVS-SC members verify that the new algorithm reduces or removes the systematic offsets while not adversely affecting other pipeline outputs. \\
Step 6. The changes are merged into the Science Pipelines, and a metric is defined to monitor the issue in future pipeline and DRP releases. \\
Step 7. The CST member documents the mitigation process and demonstrates its outcome in a public-facing document (e.g., a DM Tech Note or similar), and the issue is closed.

\textbf{Alternative success scenario} \\
Altered Step 5.1: Relevant CST and/or TVS-SC members test the new algorithm to assess whether it improves the formerly low detection efficiencies in certain types of host galaxies. 
During this verification, it is discovered that the changes that were made (e.g., lowering detection thresholds near resolved galaxies) adversely affected other scientific data products. \\
Altered Step 5.2: It is decided that it is not acceptable to sacrifice the data quality of other pipeline products to solve this issue with galactic transients. 
As an alternative, a method is devised that does not require new or altered measurements, but instead uses existing measurements such as offset from the nearest galaxy, color, morphology, etc. to derive a classification scheme that more accurately captures the subtlety of identifying transients in galaxy nuclei. \\

\textbf{Failure scenario(s)} \\
A solution is not identified, or it is determined that implementing a solution would compromise other aspects of data quality. \\
The systematic offset in nuclear transient occurrence rates with galaxy host types remains unresolved. \\

\textbf{Risks} \\
Persistent issues with specific scientific use cases may reduce community confidence in Rubin Observatory data products. \\
Alterations to the pipelines could unintentionally degrade other data products if not properly tested. \\
A lack of sufficient monitoring metrics could result in the issue re-emerging in future data releases.\\

\textbf{Related information (optional)} \\
\textbf{Priority:} Not urgent, but it is important to address emergent scientific issues.
The level of effort required depends on the frequency of such issues. \\
\textbf{Performance Target:} 1-3 months. \\
\textbf{Frequency:} 4-8 times per year (likely clustered after data releases). \\

\textbf{Preconditions for success} \\
A Data Release has been made, and sufficient time has passed for scientists to conduct research and publish findings. \\
The CST, DRP team, and relevant science collaborations are ready to address emergent issues. \\
Communication channels (e.g., email, Slack) are established for efficient coordination.
