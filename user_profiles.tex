\subsection{Students}

\subsubsection{Unsupervised}

\textbf{Description:} 
An unsupervised student user (or ``unaccompanied undergraduate") might be trying
to get their own research experience on their own; perhaps they were unable to find
a local advisor, or are getting research experience in order to apply to graduate school.
Alternatively, they might have an advisor who has given them a project outline and pointed
them at the Rubin resources to self-onboard, but who is not walking them through
the learning experience and may have little Rubin experience themself.
Students in this profile might be looking to change fields into astronomy.

\textbf{Experience:}
None to a few astronomy undergraduate courses.
None to some experience with Python and JupyterLab.
Probably no experience with Portal- or API-like interfaces.
Might have some experience in a related field like physics or computer/data science.

\textbf{Needs:}
Beginner-level tutorials that demonstrate basic coding and astronomy concepts.
Documentation with links to basic astronomy explanations.
Ideas for what kinds of analysis to do with the Rubin data.
They might strongly prefer a way to get help from their peers.
Professional development resources and guidance on paper-writing.

\subsubsection{Supervised}

\textbf{Description:} 
Supervised undergrad and graduate students with advisors that are well-versed
in the Rubin data products and services.
They are working on an analysis for a well-defined project that will yield publishable results.

\textbf{Experience:} 
A few too many astronomy courses at the upper-undergrad and graduate levels.
Currently enrolled in a university or college astronomy program.
Has experience with Python and JupyterLab.
Might have experience with Portal- or API-like interfaces.

\textbf{Needs:}
Tutorials and documentation at all levels that are specific to LSST data access and analysis.
They might prefer a way to get help from their peers.
Professional development resources and guidance on paper-writing.

\subsection{Professional scientists}

The following use profiles are all variations on the profile of an
active, publishing astronomer.
This includes postdoctoral fellows, research scientists, faculty, and
retired professionals. 

\subsubsection{Occasional user}

\textbf{Description:}
Astronomers whose main area of research is not necessarily ground-based
optical astronomy, but they're looking for LSST data to augment other data.
They are querying for LSST data for tens to hundreds of catalog objects, a few times 
a year or less.

\textbf{Experience:}
They might have limited ADQL experience and little exposure to the
basics of ground-based optical photometry measurements and errors.

\textbf{Needs:}
To TAP-query and download their small subset of table data.
To remote-query (API) for cross-matches to their objects of interest.
Little storage space and limited computational resources.
Clear table schema and a variety of ADQL recipes with descriptions.

\subsubsection{Moderate user}

\textbf{Description:}
Astronomers who frequently use ground-based optical astronomy data, either on its own
or together with other data.
They are querying LSST data for thousands to millions of catalog objects and/or
interacting with the images.
They use the RSP regularly, logging in at least once a month to work on their
ongoing projects.
They are probably working in small groups.

\textbf{Experience:}
They are experienced with the RSP and Python, and have a good general understanding
of the Rubin LSST data products.

\textbf{Needs:}
Moderate storage space and compute resources for analysis of catalog and image data.
Intermediate- and advanced-level tutorials of RSP functionality (butler, Firefly).
Creation of paper-ready data visualizations.

\subsubsection{Heavy user}

\textbf{Description:} 
Astronomers who perhaps solely use the Rubin LSST data for their research.
They are querying millions to billions of catalog objects and interacting
with images at high volume, including image reprocessing.
The RSP is their main venue for all of their LSST analysis.
They are working in small to large groups or collaborations.

\textbf{Experience:} 
They are experienced with the RSP, Python, and the LSST Science Pipelines.
They have a deep understanding of the Rubin system and its data products.

\textbf{Needs:}
A large amount of storage space that is also accessible to their collaborators.
A large amount of computational resources for image reprocessing and the
creation of user-generated data products.
Advanced-level tutorials of RSP and Science Pipelines functionality.
Creation of paper-ready data visualizations, and publishing derived data products.

\subsection{Users with disabilities}

These user profiles would be intersectional with one of the profiles above.

\subsubsection{Vision-impaired user}

\textbf{Description:}
This includes colorblind, low-vision, and blind users.
Anyone with a low visual acuity that impacts their ability to use the 
graphical user interfaces of the RSP.
The most common colorblindness is to be unable to differentiate between red and green.

\textbf{Needs:}
Rubin resources that use high-contrast colors and colorblind-friendly plots.
This applies to, e.g., default syntax highlighting, user interfaces, and tutorials.
Documentation, tutorials, and data interfaces that work well with custom software such
as screenreaders.
Data sonification code packages and the ability to generate sounds from the RSP.

\subsubsection{Hearing-impaired user}

\textbf{Description:}
Users with partial or no hearing.

\textbf{Needs:}
Written transcripts for recorded presentations.
A way to get support via a text-based interface. 

\subsubsection{Physically-impaired user}

\textbf{Description:}
Anyone who interacts with the RSP by speech, or with a single tool
(e.g., mouse or keyboard only).

\textbf{Needs:}
Interfaces that can be navigated with voice commands, or mouse- or keyboard-only.

\subsubsection{Neurodivergent users}

\textbf{Description:}
This includes users with, e.g., autism, ADHD, dyslexia.
Also includes users with social anxiety.

\textbf{Needs:}
Dyslexia-friendly fonts, uncluttered interfaces.
Documentation written in short, clear sentences and arranged in short paragraphs.
Confidential support interfaces and one-on-one Q\&A opportunities.
