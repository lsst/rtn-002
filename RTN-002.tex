\documentclass[DM,lsstdraft,authoryear,toc]{lsstdoc}
% lsstdoc documentation: https://lsst-texmf.lsst.io/lsstdoc.html
\input{meta}

% Package imports go here.
%\usepackage{graphicx}
%\usepackage{url}
%\usepackage{latexsym}
%\usepackage{color}
%\usepackage{enumitem}

% Local commands go here.

%If you want glossaries
%% DO NOT EDIT - generated by /lsst-texmf/bin/generateAcronyms.py from https://lsst-texmf.lsst.io/.
\newacronym{2D} {2D} {Two-dimensional}
\newacronym{AP} {AP} {\gls{Alert Production}}
\newacronym{AURA} {AURA} {\gls{Association of Universities for Research in Astronomy}}
\newglossaryentry{Alert} {name={Alert}, description={A packet of information for each source detected with signal-to-noise ratio > 5 in a difference image by Alert Production, containing measurement and characterization parameters based on the past 12 months of LSST observations plus small cutouts of the single-visit, template, and difference images, distributed via the internet}}
\newglossaryentry{Alert Production} {name={Alert Production}, description={Executing on the Prompt Processing system, the Alert Production payload processes and calibrates incoming images, performs Difference Image Analysis to identify DIASources and DIAObjects, and then packages the resulting alerts for distribution.}}
\newglossaryentry{Archive} {name={Archive}, description={The repository for documents required by the NSF to be kept. These include documents related to design and development, construction, integration, test, and operations of the LSST observatory system. The archive is maintained using the enterprise content management system DocuShare, which is accessible through a link on the project website www.project.lsst.org}}
\newglossaryentry{Archive Center} {name={Archive Center}, description={Part of the LSST Data Management System, the LSST archive center is a data center at NCSA that hosts the LSST Archive, which includes released science data and metadata, observatory and engineering data, and supporting software such as the LSST Software Stack}}
\newglossaryentry{Association Pipeline} {name={Association Pipeline}, description={An application that matches detected Sources or DIASources or generated Objects to an existing catalog of Objects, producing a (possibly many-to-many) set of associations and a list of unassociated inputs. Association Pipelines are used in Alert Production after DIASource generation and in the final stages of Data Release processing to ensure continuity of Object identifiers}}
\newglossaryentry{Association of Universities for Research in Astronomy} {name={Association of Universities for Research in Astronomy}, description={ consortium of US institutions and international affiliates that operates world-class astronomical observatories, AURA is the legal entity responsible for managing what it calls independent operating Centers, including LSST, under respective cooperative agreements with the National Science Foundation. AURA assumes fiducial responsibility for the funds provided through those cooperative agreements. AURA also is the legal owner of the AURA Observatory properties in Chile}}
\newglossaryentry{Butler} {name={Butler}, description={A middleware component for persisting and retrieving image datasets (raw or processed), calibration reference data, and catalogs}}
\newacronym{CCD} {CCD} {\gls{Charge-Coupled Device}}
\newacronym{CST} {CST} {Community Science Team}
\newglossaryentry{Camera} {name={Camera}, description={The LSST subsystem responsible for the 3.2-gigapixel LSST camera, which will take more than 800 panoramic images of the sky every night. SLAC leads a consortium of Department of Energy laboratories to design and build the camera sensors, optics, electronics, cryostat, filters and filter exchange mechanism, and camera control system}}
\newglossaryentry{Center} {name={Center}, description={An entity managed by AURA that is responsible for execution of a federally funded project}}
\newglossaryentry{Channel} {name={Channel}, description={An amplifier on an LSST camera CCD (see sensor). For LSST there are 16 amplifiers for each science sensor, resulting in 16 parallel data channels from each device. The 16 channels comprising a sensor are numbered from "0,0" through '1,7'.  This term may also refer to the raw data from a read-out amplifier of a sensor}}
\newglossaryentry{Charge-Coupled Device} {name={Charge-Coupled Device}, description={a particular kind of solid-state sensor for detecting optical-band photons. It is composed of a 2-D array of pixels, and one or more read-out amplifiers}}
\newglossaryentry{Commissioning} {name={Commissioning}, description={A two-year phase at the end of the Construction project during which a technical team a) integrates the various technical components of the three subsystems; b) shows their compliance with ICDs and system-level requirements as detailed in the LSST Observatory System Specifications document (OSS, LSE-30); and c) performs science verification to show compliance with the survey performance specifications as detailed in the LSST Science Requirements Document (SRD, LPM-17)}}
\newglossaryentry{Construction} {name={Construction}, description={The period during which LSST observatory facilities, components, hardware, and software are built, tested, integrated, and commissioned. Construction follows design and development and precedes operations. The LSST construction phase is funded through the NSF MREFC account}}
\newacronym{DCR} {DCR} {\gls{Differential Chromatic Refraction}}
\newacronym{DESC} {DESC} {Dark Energy \gls{Science Collaboration}}
\newacronym{DIA} {DIA} {\gls{Difference Image Analysis}}
\newglossaryentry{DIAObject} {name={DIAObject}, description={A DIAObject is the association of DIASources, by coordinate, that have been detected with signal-to-noise ratio greater than 5 in at least one difference image. It is distinguished from a regular Object in that its brightness varies in time, and from a SSObject in that it is stationary (non-moving)}}
\newglossaryentry{DIASource} {name={DIASource}, description={A DIASource is a detection with signal-to-noise ratio greater than 5 in a difference image}}
\newacronym{DM} {DM} {\gls{Data Management}}
\newacronym{DM-SST} {DM-SST} {DM System Science Team}
\newacronym{DMS} {DMS} {\gls{Data Management Subsystem}}
\newacronym{DOE} {DOE} {\gls{Department of Energy}}
\newacronym{DP} {DP} {Data Production}
\newacronym{DR} {DR} {\gls{Data Release}}
\newacronym{DRP} {DRP} {\gls{Data Release Production}}
\newglossaryentry{Data Management} {name={Data Management}, description={The LSST Subsystem responsible for the Data Management System (DMS), which will capture, store, catalog, and serve the LSST dataset to the scientific community and public. The DM team is responsible for the DMS architecture, applications, middleware, infrastructure, algorithms, and Observatory Network Design. DM is a distributed team working at LSST and partner institutions, with the DM Subsystem Manager located at LSST headquarters in Tucson}}
\newglossaryentry{Data Management Subsystem} {name={Data Management Subsystem}, description={The Data Management Subsystem is one of the four subsystems which constitute the LSST Construction Project. The Data Management Subsystem is responsible for developing and delivering the LSST Data Management System to the LSST Operations Project}}
\newglossaryentry{Data Management System} {name={Data Management System}, description={The computing infrastructure, middleware, and applications that process, store, and enable information extraction from the LSST dataset; the DMS will process peta-scale data volume, convert raw images into a faithful representation of the universe, and archive the results in a useful form. The infrastructure layer consists of the computing, storage, networking hardware, and system software. The middleware layer handles distributed processing, data access, user interface, and system operations services. The applications layer includes the data pipelines and the science data archives' products and services}}
\newglossaryentry{Data Product} {name={Data Product}, description={The LSST survey will produce three categories of Data Products. Prompt, Data Release, User Generated. Previously referred to as Levels 1, 2, and 3}}
\newglossaryentry{Data Release} {name={Data Release}, description={The approximately annual reprocessing of all LSST data, and the installation of the resulting data products in the LSST Data Access Centers, which marks the start of the two-year proprietary period}}
\newglossaryentry{Data Release Production} {name={Data Release Production}, description={An episode of (re)processing all of the accumulated LSST images, during which all output DR data products are generated. These episodes are planned to occur annually during the LSST survey, and the processing will be executed at the Archive Center. This includes Difference Imaging Analysis, generating deep Coadd Images, Source detection and association, creating Object and Solar System Object catalogs, and related metadata}}
\newglossaryentry{Department of Energy} {name={Department of Energy}, description={cabinet department of the United States federal government; the DOE has assumed technical and financial responsibility for providing the LSST camera. The DOE's responsibilities are executed by a collaboration led by SLAC National Accelerator Laboratory}}
\newglossaryentry{Difference Image} {name={Difference Image}, description={Refers to the result formed from the pixel-by-pixel difference of two images of the sky, after warping to the same pixel grid, scaling to the same photometric response, matching to the same PSF shape, and applying a correction for Differential Chromatic Refraction. The pixels in a difference thus formed should be zero (apart from noise) except for sources that are new, or have changed in brightness or position. In the LSST context, the difference is generally taken between a visit image and template. }}
\newglossaryentry{Difference Image Analysis} {name={Difference Image Analysis}, description={The detection and characterization of sources in the Difference Image that are above a configurable threshold, done as part of Alert Generation Pipeline}}
\newglossaryentry{Differential Chromatic Refraction} {name={Differential Chromatic Refraction}, description={The refraction of incident light by Earth's atmosphere causes the apparent position of objects to be shifted, and the size of this shift depends on both the wavelength of the source and its airmass at the time of observation. DCR corrections are done as a part of DIA}}
\newglossaryentry{DocuShare} {name={DocuShare}, description={The trade name for the enterprise management software used by LSST to archive and manage documents}}
\newglossaryentry{Document} {name={Document}, description={Any object (in any application supported by DocuShare or design archives such as PDMWorks or GIT) that supports project management or records milestones and deliverables of the LSST Project}}
\newacronym{FITS} {FITS} {\gls{Flexible Image Transport System}}
\newglossaryentry{Flexible Image Transport System} {name={Flexible Image Transport System}, description={an international standard in astronomy for storing images, tables, and metadata in disk files. See the IAU FITS Standard for details}}
\newglossaryentry{Handle} {name={Handle}, description={The unique identifier assigned to a document uploaded to DocuShare}}
\newacronym{IAU} {IAU} {International Astronomical Union}
\newglossaryentry{JIRA} {name={JIRA}, description={issue tracking product (not an acronym but a truncation of Gojira the Japanese name for Godzilla)}}
\newacronym{LPM} {LPM} {LSST Project Management (Document \gls{Handle})}
\newacronym{LSE} {LSE} {LSST \gls{Systems Engineering} (Document Handle)}
\newacronym{LSST} {LSST} {Legacy Survey of Space and Time (formerly Large Synoptic Survey Telescope)}
\newglossaryentry{LSST Project Office} {name={LSST Project Office}, description={Official name of the stand-alone AURA operating center responsible for execution of the LSST construction project under the NSF MREFC account}}
\newacronym{LSSTPO} {LSSTPO} {\gls{LSST Project Office}}
\newacronym{MBSE} {MBSE} {Model Based \gls{Systems Engineering}}
\newacronym{MREFC} {MREFC} {\gls{Major Research Equipment and Facility Construction}}
\newglossaryentry{Major Research Equipment and Facility Construction} {name={Major Research Equipment and Facility Construction}, description={the NSF account through which large facilities construction projects such as LSST are funded}}
\newacronym{NCSA} {NCSA} {National \gls{Center} for Supercomputing Applications}
\newacronym{NSF} {NSF} {\gls{National Science Foundation}}
\newglossaryentry{National Science Foundation} {name={National Science Foundation}, description={primary federal agency supporting research in all fields of fundamental science and engineering; NSF selects and funds projects through competitive, merit-based review}}
\newacronym{OCS} {OCS} {Observatory Control System}
\newacronym{OS} {OS} {Operating System}
\newacronym{OSS} {OSS} {Observatory System Specifications; \gls{LSE}-30}
\newglossaryentry{Object} {name={Object}, description={In LSST nomenclature this refers to an astronomical object, such as a star, galaxy, or other physical entity. E.g., comets, asteroids are also Objects but typically called a Moving Object or a Solar System Object (SSObject). One of the DRP data products is a table of Objects detected by LSST which can be static, or change brightness or position with time}}
\newacronym{OpSim} {OpSim} {\gls{Operations Simulation}}
\newglossaryentry{Operations} {name={Operations}, description={The 10-year period following construction and commissioning during which the LSST Observatory conducts its survey}}
\newglossaryentry{Operations Simulation} {name={Operations Simulation}, description={OpSim uses a sophisticated model to simulate 10 years of LSST operations using realistic seeing distributions, historical weather data, scheduled engineering downtime, and the most current telescope, dome, and camera design parameters. Under the direction of the Systems Engineering group, the OpSim group also works closely with the Telescope and Site group to ensure coordination with the OCS Scheduler development}}
\newacronym{PSF} {PSF} {Point Spread Function}
\newglossaryentry{Project Manager} {name={Project Manager}, description={The person responsible for exercising leadership and oversight over the entire LSST project; he or she controls schedule, budget, and all contingency funds}}
\newglossaryentry{Prompt Processing} {name={Prompt Processing}, description={The data processing which occurs at the Archive Center based on the stream of images coming from the telescope. This includes both Alert Production, which scans the image stream to identify and send alerts on transient and variable sources, and Solar System Processing, which identifies and characterizes objects in our solar system. It also includes specialized rapid calibration and Commissioning processing. Prompt Processing generates the Prompt Data Products.}}
\newglossaryentry{RSP} {name={RSP}, description={Rubin Science Platform}}
%\newacronym{RTN} {RTN} {Rubin Tech Note}
\newglossaryentry{Raft} {name={Raft}, description={The sensors in the LSST camera are packaged into replaceable electronic assemblies, called rafts, consisting of 9 butted sensors (CCDs) in a 3x3 mosaic. Each raft is a replaceable unit in the LSST camera. There are 21 science rafts in the camera plus 4 additional corner rafts with specialized, non-science sensors, making for a total of 189 CCDs per focal plane image. The 21 science rafts are numbered from "0,1" through "0,3", "1,0" through "3,4", and "4,1" through "4,3". (In other words, the 25 combinations from "0,0" through "4,4" minus the four corners which are non-science.)}}
\newglossaryentry{Review} {name={Review}, description={Programmatic and/or technical audits of a given component of the project, where a preferably independent committee advises further project decisions, based on the current status and their evaluation of it. The reviews assess technical performance and maturity, as well as the compliance of the design and end product with the stated requirements and interfaces}}
\newacronym{SC} {SC} {System \gls{Commissioning}}
\newacronym{SCOC} {SCOC} {Survey Cadence Optimization Committee}
\newacronym{SLAC} {SLAC} {\gls{SLAC National Accelerator Laboratory}}
\newglossaryentry{SLAC National Accelerator Laboratory} {name={SLAC National Accelerator Laboratory}, description={ A national laboratory funded by the US Department of Energy (DOE); SLAC leads a consortium of DOE laboratories that has assumed responsibility for providing the LSST camera. Although the Camera project manages its own schedule and budget, including contingency, the Camera team’s schedule and requirements are integrated with the larger Project.  The camera effort is accountable to the LSSTPO.}}
\newacronym{SP} {SP} {System Performance}
\newacronym{SRD} {SRD} {LSST Science Requirements; \gls{LPM}-17}
\newacronym{SST} {SST} {\gls{Simonyi Survey Telescope}}
\newglossaryentry{Science Collaboration} {name={Science Collaboration}, description={An autonomous body of scientists interested in a particular area of science enabled by the LSST dataset, which through precursor studies, simulations, and algorithm development lays the groundwork for the large-scale science projects the LSST will enable.  In addition to preparing their members to take full advantage of LSST early in its operations phase, the science collaborations have helped to define the system's science requirements, refine and promote the science case, and quality check design and development work}}
\newglossaryentry{Science Pipelines} {name={Science Pipelines}, description={The library of software components and the algorithms and processing pipelines assembled from them that are being developed by DM to generate science-ready data products from LSST images. The Pipelines may be executed at scale as part of LSST Prompt or Data Release processing, or pieces of them may be used in a standalone mode or executed through the LSST Science Platform. The Science Pipelines are one component of the LSST Software Stack}}
\newglossaryentry{Science Platform} {name={Science Platform}, description={A set of integrated web applications and services deployed at the LSST Data Access Centers (DACs) through which the scientific community will access, visualize, and perform next-to-the-data analysis of the LSST data products}}
\newglossaryentry{Scope} {name={Scope}, description={The work needed to be accomplished in order to deliver the product, service, or result with the specified features and functions}}
\newglossaryentry{Sensor} {name={Sensor}, description={A sensor is a generic term for a light-sensitive detector, such as a CCD. For LSST, sensors consist of a 2-D array of roughly 4K x 4K pixels, which are mounted on a raft in a 3x3 mosaic. Each sensor is divided into 16 channels or amplifiers. The 9 sensors that make up a raft are numbered from "0,0" through "2,2"}}
\newglossaryentry{Simonyi Survey Telescope} {name={Simonyi Survey Telescope}, description={The telescope at the Rubin Observatory that will perform the LSST (this refers to all physical components: the mirror, the mount assembly, etc.).}}
\newglossaryentry{Software Stack} {name={Software Stack}, description={Often referred to as the LSST Stack, or just The Stack, it is the collection of software written by the LSST Data Management Team to process, generate, and serve LSST images, transient alerts, and catalogs. The Stack includes the LSST Science Pipelines, as well as packages upon which the DM software depends. It is open source and publicly available}}
\newglossaryentry{Solar System Object} {name={Solar System Object}, description={A solar system object is an astrophysical object that is identified as part of the Solar System: planets and their satellites, asteroids, comets, etc. This class of object had historically been referred to within the LSST Project as Moving Objects}}
\newglossaryentry{Solar System Processing} {name={Solar System Processing}, description={A component of the Prompt Processing system, Solar System Processing identifies new SSObjects using unassociated DIASources.}}
\newglossaryentry{Source} {name={Source}, description={A single detection of an astrophysical object in an image, the characteristics for which are stored in the Source Catalog of the DRP database. The association of Sources that are non-moving lead to Objects; the association of moving Sources leads to Solar System Objects. (Note that in non-LSST usage "source" is often used for what LSST calls an Object.)}}
\newglossaryentry{Subsystem} {name={Subsystem}, description={A set of elements comprising a system within the larger LSST system that is responsible for a key technical deliverable of the project}}
\newglossaryentry{Subsystem Manager} {name={Subsystem Manager}, description={responsible manager for an LSST subsystem; he or she exercises authority, within prescribed limits and under scrutiny of the Project Manager, over the relevant subsystem's cost, schedule, and work plans}}
\newglossaryentry{Systems Engineering} {name={Systems Engineering}, description={an interdisciplinary field of engineering that focuses on how to design and manage complex engineering systems over their life cycles. Issues such as requirements engineering, reliability, logistics, coordination of different teams, testing and evaluation, maintainability and many other disciplines necessary for successful system development, design, implementation, and ultimate decommission become more difficult when dealing with large or complex projects. Systems engineering deals with work-processes, optimization methods, and risk management tools in such projects. It overlaps technical and human-centered disciplines such as industrial engineering, control engineering, software engineering, organizational studies, and project management. Systems engineering ensures that all likely aspects of a project or system are considered, and integrated into a whole}}
\newacronym{TBD} {TBD} {To Be Defined (Determined)}
\newacronym{TVS} {TVS} {Transients and Variable Stars \gls{Science Collaboration}}
\newglossaryentry{Task} {name={Task}, description={Tasks are the basic unit of code re-use in the LSST Stack. They perform a well defined, logically contained piece of functionality. Tasks come standard with configuration, logging, processing metadata, and debugging features. For further details, see How to Write a Task in the source code documentation.  Tasks can be nested, providing a natural way to structure - and configure - high level algorithms that delegate work to lower-level algorithms}}
\newglossaryentry{Telescope and Site} {name={Telescope and Site}, description={The LSST subsystem responsible for design and construction of the telescope structure, telescope mirrors, optical wavefront measurement and control system, telescope and observatory control systems software, and the summit and base facilities.}}
\newglossaryentry{Template} {name={Template}, description={A co-added, single-band image of the sky that is deep, and created in a manner to remove transient or fast moving objects from the final image. Constituent images for templates may be selected from a limited range of quality parameters, such as PSF size or airmass. Such images are used to perform Difference Image Analysis in order to detect variable, transient, and Solar System astrophysical objects}}
\newacronym{US} {US} {United States}
\newglossaryentry{Validation} {name={Validation}, description={A process of confirming that the delivered system will provide its desired functionality; overall, a validation process includes the evaluation, integration, and test activities carried out at the system level to ensure that the final developed system satisfies the intent and performance of that system in operations}}
\newglossaryentry{Verification} {name={Verification}, description={The process of evaluating the design, including hardware and software - to ensure the requirements have been met;  verification (of requirements) is performed by test, analysis, inspection, and/or demonstration}}
\newglossaryentry{aggregate metric} {name={aggregate metric}, description={An aggregation of multiple point metrics. For example, the overall photometric repeatability for a particular tract given given the repeatability of multiple individual stars in the tract. See also: “metric”}}
\newglossaryentry{aggregation} {name={aggregation}, description={The process of reducing multiple input values to a single output, e.g., a metric value, computed from a collection of input values. For example, a sum or average of a metric computed over patches to produce an aggregate metric at tract level. See also: “metric”, “aggregate metric”}}
\newglossaryentry{airmass} {name={airmass}, description={The pathlength of light from an astrophysical source through the Earth's atmosphere. It is given approximately by sec z, where z is the angular distance from the zenith (the point directly overhead, where airmass = 1.0) to the source}}
\newglossaryentry{algorithm} {name={algorithm}, description={A computational implementation of a calculation or some method of processing}}
\newglossaryentry{astronomical object} {name={astronomical object}, description={A star, galaxy, asteroid, or other physical object of astronomical interest. Beware: in non-LSST usage, these are often known as sources}}
\newglossaryentry{background} {name={background}, description={In an image, the background consists of contributions from the sky (e.g., clouds or scattered moonlight), and from the telescope and camera optics, which must be distinguished from the astrophysical background. The sky and instrumental backgrounds are characterized and removed by the LSST processing software using a low-order spatial function whose coefficients are recorded in the image metadata}}
\newglossaryentry{cadence} {name={cadence}, description={The sequence of pointings, visit exposures, and exposure durations performed over the course of a survey}}
\newglossaryentry{calibration} {name={calibration}, description={The process of translating signals produced by a measuring instrument such as a telescope and camera into physical units such as flux, which are used for scientific analysis. Calibration removes most of the contributions to the signal from environmental and instrumental factors, such that only the astronomical component remains}}
\newglossaryentry{camera} {name={camera}, description={An imaging device mounted at a telescope focal plane, composed of optics, a shutter, a set of filters, and one or more sensors arranged in a focal plane array}}
\newglossaryentry{configuration} {name={configuration}, description={A task-specific set of configuration parameters, also called a 'config'. The config is read-only; once a task is constructed, the same configuration will be used to process all data. This makes the data processing more predictable: it does not depend on the order in which items of data are processed. This is distinct from arguments or options, which are allowed to vary from one task invocation to the next}}
\newglossaryentry{flux} {name={flux}, description={Shorthand for radiative flux, it is a measure of the transport of radiant energy per unit area per unit time. In astronomy this is usually expressed in cgs units: erg/cm2/s}}
\newglossaryentry{metadata} {name={metadata}, description={General term for data about data, e.g., attributes of astronomical objects (e.g. images, sources, astroObjects, etc.) that are characteristics of the objects themselves, and facilitate the organization, preservation, and query of data sets. (E.g., a FITS header contains metadata)}}
\newglossaryentry{metric} {name={metric}, description={A measurable quantity which may be tracked. A metric has a name, description, unit, references, and tags (which are used for grouping). A metric is a scalar by definition. See also: aggregate metric, model metric, point metric}}
\newglossaryentry{metric value} {name={metric value}, description={The result of computing a particular metric on some given data. Note that metric values are typically computed rather than measured. See also: metric}}
\newglossaryentry{middleware} {name={middleware}, description={Software that acts as a bridge between other systems or software usually a database or network. Specifically in the Data Management System this refers to Butler for data access and Workflow management for distributed processing.}}
\newglossaryentry{model metric} {name={model metric}, description={A metric describing a model related to the data. For example, the coefficients of a 2D polynomial fit to the background of a single CCD exposure}}
\newglossaryentry{passband} {name={passband}, description={The window of wavelength or the energy range admitted by an optical system; specifically the transmission as a function of wavelength or energy. Typically the passband is limited by a filter. The width of the passband may be characterized in a variety of ways, including the width of the half-power points of the transmission curve, or by the equivalent width of a filter with 100\% transmission within the passband, and zero elsewhere}}
\newacronym{photo-z} {photo-z} {\gls{photometric redshift}}
\newglossaryentry{photometric redshift} {name={photometric redshift}, description={Often abbreviated to photo-z, this is an estimate of the true redshift (of a galaxy) determined from multi-band photometry. Generally determined from a fit of source colors to grid of model SEDs with redshift}}
\newglossaryentry{pipeline} {name={pipeline}, description={A configured sequence of software tasks (Stages) to process data and generate data products. Example: Association Pipeline}}
\newglossaryentry{point metric} {name={point metric}, description={A metric that is associated with a single entry in a catalog. Examples include the shape of a source, the standard deviation of the flux of an object detected on a Coadd, the flux of an source detected on a difference image}}
\newglossaryentry{seeing} {name={seeing}, description={An astronomical term for characterizing the stability of the atmosphere, as measured by the width of the point-spread function on images. The PSF width is also affected by a number of other factors, including the airmass, passband, and the telescope and camera optics}}
\newglossaryentry{shape} {name={shape}, description={In reference to a Source or Object, the shape is a functional characterization of its spatial intensity distribution, and the integral of the shape is the flux. Shape characterizations are a data product in the DIASource, DIAObject, Source, and Object catalogs}}
\newglossaryentry{sky map} {name={sky map}, description={A sky tessellation for LSST. The Stack includes software to define a geometric mapping from the representation of World Coordinates in input images to the LSST sky map. This tessellation is comprised of individual tracts which are, in turn, comprised of patches}}
\newglossaryentry{software} {name={software}, description={The programs and other operating information used by a computer.}}
\newglossaryentry{tract} {name={tract}, description={A portion of sky, a spherical convex polygon, within the LSST all-sky tessellation (sky map). Each tract is subdivided into sky patches}}
\newglossaryentry{transient} {name={transient}, description={A transient source is one that has been detected on a difference image, but has not been associated with either an astronomical object or a solar system body}}

%\makeglossaries

\title{Community Science Use Cases}

% Optional subtitle
% \setDocSubtitle{A subtitle}

\author{%
Melissa Graham, Andrés A. Plazas Malagón, Jeff Carlin, Leanne Guy, Christina Adair, Greg Madejski
}

\setDocRef{RTN-002}
\setDocUpstreamLocation{\url{https://github.com/rubin-observatory/rtn-002}}

\date{\vcsDate}

% Optional: name of the document's curator
% \setDocCurator{The Curator of this Document}

\setDocAbstract{%
User profiles and analysis and support use cases (workflows) for Rubin community science.
}

% Change history defined here.
% Order: oldest first.
% Fields: VERSION, DATE, DESCRIPTION, OWNER NAME.
% See LPM-51 for version number policy.
\setDocChangeRecord{%
  \addtohist{1}{2021-03-05}{Released.}{Melissa Graham}
  \addtohist{2}{2023-04-19}{Add user profiles.}{Andrés A. Plazas Malagón}
  \addtohist{3}{2024-12-01}{Reorganize document (SP-1758).}{Melissa Graham}
}

\begin{document}

% Create the title page.
\maketitle
% Frequently for a technote we do not want a title page  uncomment this to remove the title page and changelog.
% use \mkshorttitle to remove the extra pages

% ADD CONTENT HERE
% You can also use the \input command to include several content files.

\section{Introduction}

The contents of this document are under development and are subject to change.

The Community Science Team (CST) in the Rubin System Performance department maintains this set of user profiles and use cases
to capture the range of anticipated needs regarding scientific analyses with the Rubin software and data products.

This document guides the developement of a process to scientifically validate user-facing resources for scientific analysis,
such as the Rubin Science Platform, documentation, tutorials, and activities.
To "scientifically validate" the RSP means to confirm that the RSP's functionality meets the scientific needs
of its users, and that users are able to do their science with the RSP's tools.
This document also guides the development of Rubin's Model for Community Science \citedsp{RTN-004}.

\textbf{User profiles} are designed to represent the diverse user community of the RSP:
different experience and expertise backgrounds, scientific motivations, and accessibility needs.

\textbf{Analysis use cases} are designed to represent the variety of types of analysis that
users will do with the RSP, such as subsetting, visualization, model fitting, and reprocessing.
These use cases help to define the limitations of the RSP as well, and identify analysis
that requires external resources (independent data access centers; IDACs).

\textbf{Support use cases} are designed to provide workflow examples of Rubin's issue resolution
process, to illuminate how users report problems and get support, and how staff from across
Rubin participate in user support.
These uses cases include, e.g., trouble with scientific analysis,
instrumentation faults, software bugs.

\clearpage
\section{User Profiles}

\subsection{Students}

\subsubsection{Unsupervised}

\textbf{Description:} 
A student working independently (or "unaccompanied undergraduate") might be seeking to gain research experience on their own. This could be due to challenges in finding a local advisor or a desire to build research skills in preparation for applying to graduate school.
Alternatively, they might have an advisor who has given them a project outline and pointed
them at the Rubin resources to self-onboard, but who is not walking them through
the learning experience and may have little Rubin experience themself.
Students in this profile might be looking to change fields into astronomy.

\textbf{Experience:}
None to a few astronomy undergraduate courses.
None to some experience with Python and JupyterLab.
Probably no experience with Portal- or Application Programming Interface (API)-like interfaces.
Might have some experience in a related field like physics or computer/data science.

\textbf{Needs:}
Beginner-level tutorials that demonstrate basic coding and astronomy concepts.
Documentation with links to basic astronomy explanations.
Ideas for what kinds of analysis to do with the Rubin data.
They might strongly prefer a way to get help from their peers.
Professional development resources and guidance on paper-writing.

\subsubsection{Supervised}

\textbf{Description:} 
Supervised undergrad and graduate students with advisors that are well-versed
in the Rubin data products and services.
They are working on an analysis for a well-defined project that will yield publishable results.

\textbf{Experience:} 
At least a few astronomy courses at the upper-undergrad and graduate levels.
Currently enrolled in a university or college astronomy program.
Has experience with Python and JupyterLab.
Might have experience with Portal- or API-like interfaces.

\textbf{Needs:}
Tutorials and documentation at all levels that are specific to LSST data access and analysis.
They might prefer a way to get help from their peers.
Professional development resources and guidance on paper-writing.

\subsection{Professional scientists}

The following use profiles are all variations on the profile of an
active, publishing astronomer.
This includes postdoctoral fellows, research scientists, faculty, and
retired professionals. 

\subsubsection{Occasional user}

\textbf{Description:}
Astronomers whose main area of research is not necessarily ground-based
optical astronomy, but they're looking for LSST data to augment other data.
They are querying for LSST data for tens to hundreds of catalog objects, a few times 
a year or less.

\textbf{Experience:}
They might have limited Astronomical Data Query Language (ADQL) experience and little exposure to the
basics of ground-based optical photometry measurements and errors.

\textbf{Needs:}
To Table Access Protocol (TAP)-query and download their small subset of table data.
To remote-query (via API) for cross-matches to their objects of interest.
Little storage space and limited computational resources.
Clear table schema and a variety of ADQL recipes with descriptions.

\subsubsection{Moderate user}

\textbf{Description:}
Astronomers who frequently use ground-based optical astronomy data, either on its own
or together with other data.
They are querying LSST data for thousands to millions of catalog objects and/or
interacting with the images.
They use the RSP regularly, logging in at least once a month to work on their
ongoing projects.
They are probably working in small groups.

\textbf{Experience:}
They are experienced with the RSP and Python, and have a good general understanding
of the Rubin LSST data products.

\textbf{Needs:}
Moderate storage space and compute resources for analysis of catalog and image data.
Intermediate- and advanced-level tutorials of RSP functionality (butler, Firefly).
Creation of paper-ready data visualizations.

\subsubsection{Heavy user}

\textbf{Description:} 
Astronomers who perhaps solely use the Rubin LSST data for their research.
They are querying millions to billions of catalog objects and interacting
with images at high volume, including image reprocessing.
The RSP is their main venue for all of their LSST analysis.
They are working in small to large groups or collaborations.

\textbf{Experience:} 
They are experienced with the RSP, Python, and the LSST Science Pipelines.
They have a deep understanding of the Rubin system and its data products.

\textbf{Needs:}
A large amount of storage space that is also accessible to their collaborators.
A large amount of computational resources for image reprocessing and the
creation of user-generated data products.
Advanced-level tutorials of RSP and Science Pipelines functionality.
Creation of paper-ready data visualizations, and publishing derived data products.

\subsection{Users with disabilities}

These user profiles would be intersectional with one of the profiles above.

\subsubsection{Users with visual disabilities}

\textbf{Description:}
This includes colorblind, low-vision, and blind users.
Anyone with a low visual acuity that impacts their ability to use the 
graphical user interfaces of the RSP.
The most common colorblindness is to be unable to differentiate between red and green.

\textbf{Needs:}
Rubin resources that use high-contrast colors and colorblind-friendly plots.
This applies to, e.g., default syntax highlighting, user interfaces, and tutorials.
Documentation, tutorials, and data interfaces that work well with custom software such
as screenreaders.
Data sonification code packages and the ability to generate sounds from the RSP.

\subsubsection{Deaf or hard-of-hearing users}

\textbf{Description:}
Users with partial or no hearing.

\textbf{Needs:}
Written transcripts for recorded presentations.
A way to get support via a text-based interface. 

\subsubsection{Users with physical disabilities}

\textbf{Description:}
Anyone who interacts with the RSP by speech, or with a single tool
(e.g., mouse or keyboard only).

\textbf{Needs:}
Interfaces that can be navigated with voice commands, or mouse- or keyboard-only.

\subsubsection{Neurodivergent users}

\textbf{Description:}
This includes users with, e.g., autism, ADHD, dyslexia.
Also includes users with social anxiety.

\textbf{Needs:}
Dyslexia-friendly fonts, uncluttered interfaces.
Documentation written in short, clear sentences and arranged in short paragraphs.
Confidential support interfaces and one-on-one Q\&A opportunities.



\clearpage
\section{Analysis Use Cases}

These analysis use cases represent general things that users want to do in the Rubin Science Platform,
using specific scientific examples.
They are not 1:1 related to the user profiles above, although some of the more advanced analysis
use cases, for example, would be much more likely to be done by a heavy user.

The analysis use cases progress from "simple" to "more complicated".
These will serve as the basis for usability tests and the CST's process for science validation of
the RSP.

\subsection{Rubin Science Platform}


\subsubsection{Portal}

\textbf{P-101}: User interface cone search on the object catalog, review the results overplotted on the Hierarchical Progressive Survey (HiPS) map.

\textbf{P-102}: Astronomical Data Query Language (ADQL) search with column constraints, followed by the creation of plots based on the retrieved data in the results view.

\textbf{P-103}: Data discoverability via e.g., schema and HiPS browsing. 

\textbf{P-201}: Extraction of image cutouts from LSST image archives based on user-defined coordinates, spatial dimensions, and filters.

\textbf{P-202}:  View, rerun, and refine previous ADQL queries. Save, organize, and share queries for reproducibility and collaboration.

\textbf{P-203}: Upload user-defined tables.

\textbf{P-301}: Identification of LSST images that spatially and/or temporally overlap user-defined coordinates and timeframes. Enable visualization and interaction with these images, including zooming, panning, rescaling, performing pixel statistics, and adding markup. Generate and save publication-quality plots and images for analysis and presentation.

\subsubsection{Notebook}

\textbf{N-100}: Explore and analyze images by loading a specific observation using the LSST data butler. Use basic visualization tools, such as color mapping and zoom functions, to examine image details. Identify and measure properties of individual sources (e.g., stars or galaxies) within the image to enable further analysis or classification.

\textbf{N101}: Develop and utilize custom algorithms to detect and characterize specific astronomical objects, such as supernovae or quasars.

\textbf{N-102}: Reprocess data using the LSST Science Pipelines with modified configuration parameters to suit specific research needs.

\textbf{N-103}: Utilize advanced plotting libraries like Matplotlib and Bokeh to create publication-quality figures.

\textbf{N-201}: Query the LSST databases (e.g., Prompt Products Database) to generate cutout images (on the order of tens of thousands) for target candidates of specific astronomical objects of interest (e.g., lensed AGNs).

\textbf{N-202}: Implement and apply algorithms to detect and characterize transient or variable phenomena, integrating external packages for computational support when needed (e.g., to identify microlensing events).

\textbf{N-203}: Identify and analyze patterns of data quality degradation or observational interference, such as satellite streaks, and develop models to assess their impact on science (e.g., on Solar System science) and optimize observational strategies.

\textbf{N-301}: Fit light curves for selected objects by applying an initial pre-selection function to filter candidates based on user-defined criteria, such as object features or custom metrics.
Pass the light curves of the selected objects to a fitting function to obtain best-fit parameters for each object for e.g.,  detailed modeling of variable or transient events. 

\textbf{N-302}: Run and compare different object-finding algorithms (e.g., galaxy clusters), assessing their performance using custom-defined metrics. Utilize batch processing resources for computationally intensive tasks.

\subsubsection{API}

\textbf{A-101}: Explore LSST catalogs and metadata using basic API queries, and cross-match catalogs from LSST and other surveys.

\textbf{A-102}: Run ADQL queries on the LSST datasets to explore tables and produce visualizations via an interactive graphical users interface (GUI).

\textbf{A-201}: Verify the consistency of observables across multiple datasets and perform statistical checks (for example, for multiple dark energy probes). 

\textbf{A-202}: Extract photometric data for sources and apply different algorithms for deblending, variability estimation, and source property analysis (e.g., for AGN variability). 

\textbf{A-301}: Query galaxy shapes, positions, and photo-z catalogs to measure two-point correlation functions with external packages (\emph{e.g.}, Treecorr).
Use the results for modeling and parameter sampling with access to batch resources.

\textbf{A-302}: Cross-match LSST coadd data from multiple data releases with ancillary datasets, potentially incorporating single-alert matches via broker streams.
Publish all cross-match results as User Generated Data Products (UGDPs) for broad community use, including applications in Solar System science, cosmology, transients, and more.

\subsubsection{Multi-aspect}

\textbf{M-201}: Create exploratory plots and diagrams in the Portal to identify trends or interesting subsets, then use the Notebook Aspect for computationally intensive analyses and catalog queries.


\subsection{Independent Data Access Centers (IDACs)}




\clearpage
\section{Support Use Cases}

\clearpage
{\bf Use Case:} {\it unique identifier number} {\it name (short active verb phrase)} \\

{\bf Characteristic Information} \\
Trigger: {\it the action upon the system that starts the use case, may be time event} \\
Goal in Context: {\it a longer statement of the goal, if needed} \\
Primary Actor: {\it a role name for the primary actor, or description} \\
Scope: {\it what system is being considered black-box under design} \\
Level: {\it one of: Summary, Primary task, Subfunction} \\
Preconditions: {\it what we expect is already the state of the world} \\
Success End Condition: {\it the state of the world upon successful completion} \\
Failed End Condition: {\it the state of the world if goal abandoned} \\

{\bf Main Success Scenario} \\
{\it put here the steps of the scenario from trigger to goal delivery, and any cleanup after} \\
Step 1: {\it action description} \\
Step 2: {\it action description} \\
Etc.

{\bf Extensions} \\
{\it put here there extensions, one at a time, each referring to the step of the main scenario} \\
Alteration 1: IF {\it condition} THEN {\it action or sub-use case} \\
Alteration 2: IF {\it condition} THEN {\it action or sub-use case} \\
Etc.

{\bf Sub-Variations} \\
{\it put here the sub-variations that will cause eventual bifurcation in the scenario} \\
Variation 1: IF {\it condition} THEN {\it list of sub-variations} \\
Variation 2: IF {\it condition} THEN {\it list of sub-variations} \\

{\bf Related Information} (Optional) \\
Priority: {\it how critical to your system / organization} \\
Performance Target: {\it the amount of time this use case should take} \\
Frequency: {\it how often it is expected to happen} \\
Superordinate Use Case: {\it optional, name of use case that includes this one} \\
Subordinate Use Cases: {\it optional, depending on tools, links to sub.use cases} \\
Channel to primary actor: {\it e.g. interactive, static files, database} \\
Secondary Actors: {\it list of other systems needed to accomplish use case} \\
Channel to Secondary Actors: {\it e.g. interactive, static, file, database, timeout} \\

{\bf Open Issues} (Optional) \\
{\it list of issues about this use cases awaiting decisions} \\

{\bf Schedule} \\
Due Date: \\


\clearpage
\subsection{S-001: Generic community self-help using CST resources}

{\bf Issue origin and description} \\
User posts a request for help in the Forum's Support category. \\

{\bf Success scenario(s)} \\
The community helps the user with no intervention from Rubin staff. \\

{\bf Success workflow} \\
Step 1. A user makes a post in the Forum detailing their issue.\\
Step 2. Community members reply with suggestions.\\
Step 3. The user solves their issue and marks their post solved.\\

{\bf Alternative success scenario} \\
The community cannot solve the issue and the SP-CST addresses it instead. \\

{\bf Failure scenario(s)} \\
The reported issue remains unresolved. \\

{\bf Risks}\\
If the community cannot solve the issue, this creates more work for the CST. \\
If the CST misses or cannot solve the issue, use of the Forum will decrease. \\
If users are unable to get support for science, LSST will be less impactful. \\

{\bf Related information (optional) \\
Priority: High \\
Timeframe: Days \\
Frequency: High \\

{\bf Preconditions for success}\\
The Rubin Community Forum had to exist. \\
The Forum has to be used by community members. \\

\clearpage
\subsection{S-002: Generic CST-provided help}

{\bf Issue origin and description} \\
User posts a request for help with a simple camera issue in the Forum's Support category. \\

{\bf Success scenario(s)} \\
A CST member posts a solution within a couple of days. \\

{\bf Success workflow} \\
Step 1. A user makes a post in the Forum detailing their issue. \\
Step 2. A CST member responds within 24 hours to affirm the issue is being addressed. \\
Step 3. The CST member reaches out in the staff Slack space to camera team members. \\
Step 4. The CST member compiles responses from othe camera team and posts a response. \\
Step 5. The user confirms their question is answered and marks the solution. \\

{\bf Alternative success scenario} \\
A community member solves the issue before a Rubin staff member. \\
The CST member finds the answer in camera team documentation and posts it. \\

{\bf Failure scenario(s)} \\
The reported issue remains unresolved. \\

{\bf Risks}\\
If the CST misses or cannot solve the issue, use of the Forum will decrease. \\
If users are unable to get support for science, LSST will be less impactful. \\

{\bf Related information (optional) \\
Priority: High \\
Timeframe: Days \\
Frequency: High \\

{\bf Preconditions for success}\\
The Rubin Community Forum had to exist. \\
The Forum has to be used by community members. \\
The CST member had to liaise internally with camera team members (or relevant documentation had to exist). \\


\clearpage
\subsection{A Camera Fault Requires Modification of the Alert Production System}
{\bf Use Case:} 003 Observatory Personnel Detect a Fault, Leads to Change in Alert Production Pipeline \\

{\bf Characteristic Information} \\
Trigger: A fault in one of the Raft Electronics Boards (a sensor). \\
Goal in Context: This fault causes and abnormally high number of DIASource detections for that sensor, most of which are appropriately flagged as artifacts by the real/bogus algorithm and do not become Alerts, but some are not flagged and are released. \\
Primary Actor: The System Performance Verification and Validation (SP-VV) Lead Scientist \\
Scope: alert production, broker community \\
Level: \\
Preconditions: Operations \\
Success End Condition: The sensor anomaly is solved by modifying the Alert Production (AP) pipeline. \\
Failed End Condition: Bogus DIASources from this sensor fault continue to be released as Alerts. \\

{\bf Main Success Scenario} \\
Step 1: Rubin Observatory Operations team monitors verification and validation outputs during the night. \\
Step 2: Night staff notice an anomalous spike in the number of DIASources detected. \\
Step 3: Night staff run diagnostics to isolate the excess DIASources to a single sensor, identify the underlying fault, and determine that the hardware cannot be quickly/easily fixed (i.e., sensor will have to be replaced). \\
Step 4: Night staff summarize their findings for the SP-VV Lead Scientist. \\
Step 5: SP-VV team runs further diagnostics and coordinates with Data Production (DP) to plan a fix. \\
Step 6: DP implements the fix (e.g., retraining the real/bogus characterization). \\
Step 7: SP-VV and DP coordinate with the CET to summarize the issue, it's fix, and potential science impacts for the community. \\

{\bf Extensions} \\
Alteration 5.1: IF a significant amount of the anomalous DIASources were released as Alerts THEN SP-VV team coordinates with the CET to prepare an initial statement describing the fault and it's impact on the Alert stream, to be posted to the Community Forum and sent to a community brokers email list. \\

{\bf Sub-Variations} \\
Variation 3.1: IF the sensor anomaly is fixed with hardware THEN a software fix might not be necessary \\

{\bf Related Information} (Optional) \\
Priority: High  \\
Performance Target: identify scope of issue within one day; apply fix to AP within a week \\
Frequency: Low \\
Superordinate Use Case:  \\
Subordinate Use Cases: \\
Channel to primary actor: Internal discussions via Slack channels \\
Secondary Actors: CET, DP, science users \\
Channel to Secondary Actors: posts to the science users via the Community Forum \\

{\bf Open Issues} (Optional) \\

{\bf Schedule} \\
Due Date: \\


\clearpage
\subsection{Large Queries: Community Self-Help Using CST Resources}
{\bf Use Case:} 004 Large Queries: Community Self-Help Using CET Resources  \\

{\bf Characteristic Information} \\
Trigger: A user's request for help posted on the LSST Community web forum. \\
Goal in Context: A user receives assistance in executing large queries on a Rubin data product. \\
Primary Actor: The System Performance Community Engagement Team (SP-CET) \\
Scope: SP-CET \\
Level: \\
Preconditions: In-Operations, with users accessing the Rubin Science Platform Notebook Aspect, for which training tutorials exist and are served by the SP-CET \\
Success End Condition: The user has a path towards solving their issue. \\
Failed End Condition: The user does not know how to solve their issue. \\

{\bf Main Success Scenario} \\
Step 1: A user makes a post on the LSST Community web forum detailing issues they are having in querying the Object catalog from the latest data release (their Jupyter Notebook will not run.) \\
Step 2: Other users and an SP-CET member respond to the post with advice about their query. \\
Step 3: Discussion thread on the post realizes the user needs more computational resources. \\
Step 4: The user applies to the Resource Allocation Committee for additional processing. \\

{\bf Extensions} \\
Step 4 Alteration: Optional addition, the SP-CET add this user's particular case to documenation about large queries to avoid such issues in the future. \\

{\bf Sub-Variations} \\
\\

{\bf Related Information} (Optional) \\
Priority: Low  \\
Performance Target: a few days at most \\
Frequency: probably pretty often \\
Superordinate Use Case:  \\
Subordinate Use Cases: \\
Channel to primary actor: Interactive (web forum) \\
Secondary Actors: other community users \\
Channel to Secondary Actors: Interactive (web forum) \\

{\bf Open Issues} (Optional) \\

{\bf Schedule} \\
Due Date: \\


\clearpage
\subsection{Science Platform Bug: A Help Desk Submission}
\subsection{S-005: Science Platform Issue - A Help Desk Submission}

\textbf{Issue origin and description} \\
A user submits a Help Desk ticket regarding an issue they encountered in the Rubin Science Platform (RSP).
The CST is responsible for determining whether the issue is a user error or an actual bug and taking appropriate steps to resolve it.

\textbf{Success scenario(s)} \\
The ticket is resolved by either: (1) solving a user issue that is not actually a bug, or (2) identifying the bug and providing detailed information to the RSP development team.
While the ultimate goal is fixing the bug, providing a short-term workaround for the user is a critical success factor for the CST.

\textbf{Success workflow} \\
Step 1. CST confirms that the reported bug is reproducible. \\
Step 2. CST communicates the details of the bug to the RSP development team. \\
Step 3. CST provides a short-term workaround to the user, if possible. \\
Step 4. RSP developers investigate and fix the bug. \\
Step 5. A new version of the RSP, incorporating the fix, is rolled out. \\

\textbf{Alternative success scenario} \\
The CST determines that the issue is not a bug but rather a misunderstanding or error by the user.
In this case, the CST provides clarification or instructions to help the user resolve their issue.

\textbf{Failure scenario(s)} \\
The bug cannot be reproduced, and no workaround is available. \\
The issue persists without a resolution, potentially affecting the robustness of the RSP.\\

\textbf{Risks} \\
Users may lose confidence in the RSP if bugs are not addressed promptly or if no workarounds are provided. \\
The lack of timely fixes may lead to broader disruptions in the community's ability to use the platform effectively. \\
Recurrent or unresolved bugs may increase the CST and RSP development teams' workload.\\

\textbf{Related information (optional)} \\
\textbf{Priority:} Most cases are likely to be low priority. \\
\textbf{Performance Target:} Resolution within days to weeks, depending on the issue complexity. \\
\textbf{Frequency:} Infrequent \\

\textbf{Preconditions for success} \\
The RSP is operational and actively used by the community. \\
A functional ticketing system is in place to track and manage Help Desk submissions. \\
Communication channels between the CST and RSP development teams (e.g., JIRA, Slack) are active and effective. \\


\clearpage
\subsection{Survey Strategy Alteration for Photometric Redshifts}
\textbf{Use Case:} 006 Survey Strategy Alteration for Photometric Redshifts \\

\textbf{Characteristic Information} \\
\textbf{Trigger:} DESC photo-z and weak lensing working groups determine that their projected photo-z statistics fall short of the LSST SRD year 10 target values for photo-z quality, and the DESC spokesperson communicates this formally to the SP-CST lead via email. \\
\textbf{Goal in Context:} As the Rubin Observatory intends to meet the science requirements of its LSST survey, staff will need to study and implement changes to the survey strategy to acquire the necessary data to meet the photometric redshift statistics requirements. \\
\textbf{Primary Actor:} SP-CST lead, upon receipt of the email, opens an issue in an issue tracking system. \\
\textbf{Scope:} This use case defines the interaction of the survey performance team with the observatory operations team as a formal request for a change comes from the science community. \\
\textbf{Level:} Summary \\
\textbf{Preconditions:} The Observatory is in full operations, and data has been released to the science collaborations. This scenario likely takes place in year 4 of the survey, with year 3 data under study by the DESC. It assumes a typical lag of about one year for science collaborations to analyze data and provide formal feedback. Most scientists believe that limitations in photo-zs can be improved by changing the data acquisition strategy rather than algorithms. \\
\textbf{Success End Condition:} A year's worth of new data improves the photo-z statistics as expected, and the DESC photo-z and weak lensing working groups project the 10-year data to meet the requirements. \\
\textbf{Failed End Condition:} The photo-z statistics remain insufficient. The SRD photo-z requirements are not met, making the photo-z statistics the leading systematic limiting the weak lensing cosmology results. This restricts the precision of the resulting cosmological parameters. \\

\textbf{Main Success Scenario} \\
0. The DESC determines there is an issue with the photo-zs and communicates this to the SP-CST. \\
1. SP-CST creates an issue ticket assigned to the Community Scientist with expertise in cosmology, adding relevant watchers from the SP Survey Scheduling Team (SP-SST), Data Production Algorithms and Pipeline Team (DP-AP), and Observatory Operations Observatory Science Team (OO-OS). \\
2. A meeting is organized with DESC scientists, SP-CST, SP-SST, DP-AP, OO-OS representatives, and members of the Survey Cadence Optimization Committee (SCOC) to decide on next steps. It is decided to perform simulations of a cadence that increases cumulative exposure times in the u and y bands. \\
3. The SP-SST publishes a Community post describing the issue and the planned actions and responds to the DESC spokesperson. \\
4. The SP-SST generates the OpSim results with standard metrics. \\ 
5. The DESC (and other science collaborations) evaluate the simulations to determine the impact on weak lensing and photo-z issues, as well as other science cases, and submit reports back to the SCOC. \\
6. The SCOC meets (\approx 6 months) and decides to recommend a u-band focus for year 9 to the Directorate. A written summary of their discussion is provided to the SP-CST. \\
7. The SCOC written summary is made publicly available by the SP-CST. \\
8. The issue ticket is closed. \\

\textbf{Extensions} \\
\\

\textbf{Sub-Variations} \\
\\

\textbf{Related Information} (Optional) \\
\textbf{Priority:} This issue is of high priority but is not particularly urgent. \\
\textbf{Performance Target:} The time from opening to closing the ticket should be less than 1 year. \\
\textbf{Frequency:} Issues affecting the wide-field cadence are likely to arise every couple of years. \\
\textbf{Superordinate Use Case:} \\
\textbf{Subordinate Use Cases:} \\
\textbf{Channel to Primary Actor:} Interactive (email) \\
\textbf{Secondary Actors:} SP-SST, DP-AP, OO-OS, SCOC \\
\textbf{Channel to Secondary Actors:} Interactive (JIRA) \\

\textbf{Open Issues} (Optional) \\

\textbf{Schedule} \\
\textbf{Due Date:} \\


\clearpage
\subsection{A Science-Driven Modification to a Data Product}
{\bf Use Case:} 007 A Science-Driven Modification to a Data Product \\

{\bf Characteristic Information} \\
Trigger: A manuscript is published in teh literature that highlights and issue with Rubin Observatory Data Release products. This paper is sceen by a CET member, who passes it along to the Community Scientist and Science Collaboration with the relevant expertise. \\
Goal in Context: While the Data Releases will be subject to verification adn validation before release, it is inevitable that sub-optimal outcomes for certain science goals will be identified as a result of data processing decisions that were taken (or weren't even considered).  The CET and the DRP team need to implement processes to (1) assess the possible impacts of alterations to the code baes that solve the issue, and (2) if possible, incorporate code changes that mitigate the emergent scientific issue while not compromising overal DRP results. \\
Primary Actor: The CET member with primary expertise in transients. \\
Scope: Data Release Production \\
Level: \\
Preconditions: A Data Release has occurred, with sufficient time passed for scientists to carry reseach projects throught to publication. \\
Success End Condition: The DRP tema implements and tests and algorithmic change that solves the problem. \\
Failed End Condition: A solution is either not identified, or is deemed deterimental to other aspects of data quality, and the systematic offset in nuclear transient occurrence rates with galaxy host types remains unchanged. \\

{\bf Main Success Scenario} \\
1. A manuscript is published in the literature that highlights a systematic offset in the detection efficiency of nuclear transients in different types of host galazies with Rubin Observatory Data Release products. \\
2. This paper is brought to the attention of a CET member, who creates an Issue Ticket assigned to the Community Scientist(s) and Science Collaboration(s) with the relevant expertise (in this case, predominantly time-domain science and the TVS SC). \\
3. The Community Scientist and other relevant parties investigate the issue and identify a possible cause in the DM pipelines. \\
4. The ticket is discussed with DM developers, who identify possible improvements to mitigate the issue, and consider the possible side effects of implementing those changes. \\
5. DM developers implement the code changes on a ticket branch, and the relevant CET and/or TVS-SC verify that the new algoithm decreases (or removes) the systematic offsets with host galaxy type from the delivered detection efficiencies, while not adversely affecting other pipeline outputs. \\
6. The changes are merged to the science pipelines, and one of the relevant parties defines a metric that can be monitored each time a new version of the Science Pipelines and DRP is released. \\
7.  The CET member details the mitigation and demonstrates its outcome in a public-facing document (e.g., a DM Tech Note or similar), and the issue is closed. \\

{\bf Extensions} \\
Altered Step 5.1: Relevvant CET and/or TVS-SC members test the new algorithm to assess whether it improves the formerly low detection efficiencies in certain types of host galaxies.  During this verification, it is discovered that the changes that were made (e.g., lowering detection thresholds near resolved galaxies) adversely affected other scientific data products. \\
Altered Step 5.2: It is decided that it is not acceptable to sacrifice data quality of other pipeline products to solve this issue with galactic transients.  As an alternative, a methood is devised that does not require new or altered measurements, but instead uses existing measurements such as offset from the nearest galaxy, color, morphology, etc. to derive a classification scheme that more accurately captures the subtlety of identifying transients in galaxy nuclei. \\

{\bf Sub-Variations} \\
\\

{\bf Related Information} (Optional) \\
Priority: Not urgent, but important to have a process to deal with emergent science effects such as this.  The amount of effort to be spent will depend on how frequency issues like this come to our attention. \\
Performance Target: 1-3 months \\
Frequency: 4-8 times per year (most likely all clustered in a period shortly after data releases) \\
Superordinate Use Case:  \\
Subordinate Use Cases: \\
Channel to primary actor: Personal communication (email, Slack, or during regular meetings) to CET or individual Community Scientist \\
Secondary Actors: Data Management developer(s), interested Science Collaboration members \\
Channel to Secondary Actors: Email or Slack (perhaps first passing through the DM-SST) \\

{\bf Open Issues} (Optional) \\

{\bf Schedule} \\
Due Date: \\




\appendix
% Include all the relevant bib files.
% https://lsst-texmf.lsst.io/lsstdoc.html#bibliographies
\section{References} \label{sec:bib}
\renewcommand{\refname}{} % Suppress default Bibliography section
\bibliography{local,lsst,lsst-dm,refs_ads,refs,books}

% Make sure lsst-texmf/bin/generateAcronyms.py is in your path
\section{Acronyms} \label{sec:acronyms}
\addtocounter{table}{-1}
\begin{longtable}{p{0.145\textwidth}p{0.8\textwidth}}\hline
\textbf{Acronym} & \textbf{Description}  \\\hline

CET & Community Engagement Team \\\hline
DM & Data Management \\\hline
LSST & Legacy Survey of Space and Time (formerly Large Synoptic Survey Telescope) \\\hline
MBSE & Model Based Systems Engineering \\\hline
RTN & Rubin Tech Note \\\hline
SP & Survey Performance \\\hline
TBD & To Be Defined (Determined) \\\hline
\end{longtable}

% If you want glossary uncomment below -- comment out the two lines above
% \printglossaries





\end{document}
