\subsection{S-001: Generic community self-help using CST resources}

\textbf{Issue origin and description} \\
User posts a request for help in the Forum's Support category. \\

\textbf{Success scenario(s)} \\
The community helps the user with no intervention from Rubin staff. \\

\textbf{Success workflow} \\
Step 1. A user makes a post in the Forum detailing their issue. \\
Step 2. Community members reply with suggestions. \\
Step 3. The user solves their issue and marks their post solved. \\

\textbf{Alternative success scenario} \\
The community cannot solve the issue and the SP-CST addresses it instead. \\

\textbf{Failure scenario(s)} \\
The reported issue remains unresolved. \\

\textbf{Risks} \\
If the community cannot solve the issue, this creates more work for the CST. \\
If the CST misses or cannot solve the issue, use of the Forum will decrease. \\
If users are unable to get support for science, LSST will be less impactful. \\

\textbf{Related information (optional)} \\
\textbf{Priority:} High \\
\textbf{Timeframe:} Days \\
\textbf{Frequency:} High \\

\textbf{Preconditions for success} \\
The Rubin Community Forum had to exist. \\
The Forum has to be used by community members. \\
