
\subsubsection{Portal}

\textbf{P-101}: User interface cone search on the object catalog, review the results overplotted on the HIPS map.

\textbf{P-102}: Astronomical Data Query Language (ADQL) cone search on the object catalog with magnitude constraints, create a color-magnitude diagram in the results view.

\textbf{P-103}: Create exploratory color plots of different trans-Neptunian object (TNO) populations and perform statistical analyses to determine potential correlations. 

\textbf{P-201}: Perform an LSST light curve classification (e.g., stars vs quasars vs AGNs) using Portal tools, leveraging visualization features to distinguish populations.

\textbf{P-301}: Query annual data releases and deepCoadds for overlaps with other datasets, and generate publication-quality plots of data and images.

\textbf{P-302}: Query survey property maps and cross-correlate them with shape and photo-z catalogs to investigate cosmic shear. Perform advanced filtering to optimize query performance.

\subsubsection{Notebook}

\textbf{N-101}:  Analyze light curves and create periodograms for selected targets, starting with provided templates and basic functionality.

\textbf{N-201}: Perform Point Spread Function (PSF) null tests for weak lensing analyses, querying PSF moments to compute residuals in sizes and shapes, and plotting them as a function of magnitude to examine systematic effects (e.g., the brighter-fatter effect).

\textbf{N-202}: Detect and characterize microlensing events across the sky, using advanced algorithms and external packages for computational support.

\textbf{N-203}: Assess the impact of megaconstellations on Solar System science by identifying compromised observations of Solar System objects due to satellite streaks.
Develop statistical models and optimize survey strategies based on these constraints. 

\textbf{N-301}: Run and compare different cluster-finding algorithms, assessing their performance using custom-defined metrics. Utilize batch processing resources for computationally intensive tasks.

\textbf{N-302}: Characterize the orbital dynamics of solar system small bodies using LSST catalogs and Minor Planet Center data.
Refine orbits, provide orbital characterizations, and develop scalable computational tools for efficient analysis.

\subsubsection{API}

\textbf{A-101}: Explore Rubin catalogs and metadata using basic API queries, and cross-match catalogs from Rubin and other surveys to study low-redshift Type Ia supernovae.

\textbf{A-102}: Monitor known strong lensing systems and discover new ones by performing API-based queries to measure time delays of multiply-imaged transients and quasars.

\textbf{A-201}: Verify the consistency of cosmological observables across dark energy probes by querying Rubin Web API data and performing basic statistical consistency checks.

\textbf{A-202}: Characterize AGNs in LSST data by extracting photometry for extended sources. Use Rubin APIs to run deblending algorithms, estimate variability parameters, and analyze AGN activity.

\textbf{A-301}: Query galaxy shapes, positions, and photo-z catalogs to measure two-point correlation functions with external packages (\emph{e.g.}, Treecorr).
Use the results for modeling and parameter sampling with access to batch resources.

\textbf{A-302}: Cross-match LSST coadd data from multiple data releases with ancillary datasets, potentially incorporating single-alert matches via broker streams.
Publish all cross-match results as User Generated Data Products (UGDPs) for broad community use, including applications in Solar System science, cosmology, transients, and more.

\subsubsection{Multi-aspect}

\textbf{M-201}: Create exploratory plots and diagrams in the Portal to identify trends or interesting subsets, then use the Notebook Aspect for computationally intensive analyses and catalog queries.
