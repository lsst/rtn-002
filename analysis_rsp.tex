
\subsubsection{Portal}

\textbf{P-101}: User interface cone search on the object catalog, review the results overplotted on the HIPS map.

\textbf{P-102}: ADQL cone search on the object catalog with magnitude constraints, create a color-magnitude diagram in the results view.


\textbf{BELOW ARE THE ORIGINAL USER PROFILES which have lots of good analysis use-cases to incorporate here.}

\textbf{P1:} A user with little experience using portal-like websites who wants to make exploratory color plots of different TNO populations and perform statistical analyses on them to determine potential correlations. 

\textbf{P2}: An experienced user (previous experience with e.g., MAST or IRSA) who will use the portal  LSST Light curve classification (stars vs quasars vs AGNs or other objects). 

\textbf{P3}: A static science user with little portal experience looking for overlaps with other data sets by querying on annual data releases and deepCoadds. They want to produce publication-quality plots of data and images. 

\textbf{P4}: An experienced static science user that wants to determine queries for survey property maps and cross-correlate them with shape and photo-z catalogs for cosmic shear investigations. 

\subsubsection{Notebook}

\textbf{N1}: A time domain user with little experience in python and Jupyter Notebooks/Jupyter Lab, who is interested in light curves and periodograms of selected targets. 

\textbf{N2}:   A time domain user experienced with programming experience, including python and Jupyter Notebooks, who is interested in detection and characterization of microlensing events across the sky.

\textbf{N3}:   A static science user with little Jupyter experience wants to Perform Point Spread Function (PSF) null tests for weak lensing analyses: for example, they wish to query residual PSF sizes and plot them as function of magnitude to look for residuals of the brighter-fatter effect (e.g., Jarvis et al 2017).

\textbf{N4}:  An experienced static science user interested in running different cluster-finding algorithms and assessing their performances under different metrics. Might need the use of batch resources. 

\subsubsection{API}

\textbf{A1}: A time domain user with no previous experience with Astronomical Data Query Language (ADQL) queries who is interested in exploring what catalogs exist and which columns they contain for Rubin catalogs and catalogs from other surveys. They are interested in performing basic cross matching for low-redshift Type Ia supernovae studies. 

\textbf{A2}: An experienced time domain user interested in discovery and characterization of variable strong lenses for potential follow up. They also want to monitor the list of existing strong lensing systems (cross matching with other lists). Eventually, the user will be interested in measuring time delays of multiply imaged transients and quasars. 

\textbf{A3}: A static science user with little experience with the Web API aspect wants to check that measurements of cosmological observables from the various dark energy probes agree within expected uncertainties to check data consistency. 

\textbf{A4}: An experienced static science user who wants to query shape and photo-z catalogs, measure two point correlation functions with an external package (\emph{e.g.}, 11Treecorr”) to produce a data vector. Then perform modeling and parameter sampling. This might need access to extra resources (batch processing).

\subsubsection{Multi-aspect}

\textbf{M1}: A researcher who wants to first create exploratory plots and diagrams in the Portal, then perform computationally intensive calculations and catalog analyses in the Notebook Aspect. 





