\subsection{S-004: Large Queries - Community Self-Help Using CST Resources}

\textbf{Issue origin and description} \\
A user posts a request for help on the Rubin Community forum regarding issues in executing large queries on a Rubin data product.

\textbf{Success scenario(s)} \\
The user receives assistance and gains a clear path toward solving their issue, including understanding their resource needs or improving their query execution.

\textbf{Success workflow} \\
Step 1. A user makes a post on the Rubin Community forum detailing their issue (e.g., problems with querying the Object catalog from the latest data release, and their Jupyter Notebook will not run). \\
Step 2. Other users and an SP-CST member respond to the post with advice about improving the query or addressing their problem. \\
Step 3. Through the discussion thread, it is identified that the user requires additional computational resources to execute their query. \\
Step 4. The user applies to the Resource Allocation Committee for access to additional processing resources, enabling them to successfully execute their query.

\textbf{Alternative success scenario} \\
Step 4 Alteration: The SP-CST adds this particular query issue to documentation or tutorials on handling large queries to avoid similar issues in the future.

\textbf{Failure scenario(s)} \\
The user is unable to solve their issue, either due to the lack of adequate support or inability to secure the necessary resources.

\textbf{Risks} \\
Users may lose confidence in the Rubin Science Platform if they are unable to resolve their queries effectively. \\
Increased workload for the CST if such issues become frequent without proper documentation or guidance. \\
Without additional resource allocation in certain cases, users may not be able to complete large queries, impacting their scientific productivity.

\textbf{Related information (optional)} \\
\textbf{Priority:} Low \\
\textbf{Performance Target:} A few days at most \\
\textbf{Frequency:} Likely frequent \\

\textbf{Preconditions for success} \\
Users have access to the Rubin Science Platform Notebook Aspect. \\
Training tutorials for query execution in th eNotebook Aspect are available and accessible to users. \\
The Rubin Community forum is actively monitored by the SP-CST and other community members. \\
A functional Resource Allocation Committee exists for handling additional resource requests.\\
