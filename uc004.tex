{\bf Use Case:} 004 Large Queries: Community Self-Help Using CST Resources  \\

{\bf Characteristic Information} \\
Trigger: A user's request for help posted on the LSST Community web forum. \\
Goal in Context: A user receives assistance in executing large queries on a Rubin data product. \\
Primary Actor: The System Performance Community Engagement Team (SP-CST) \\
Scope: SP-CST \\
Level: \\
Preconditions: In-Operations, with users accessing the Rubin Science Platform Notebook Aspect, for which training tutorials exist and are served by the SP-CST \\
Success End Condition: The user has a path towards solving their issue. \\
Failed End Condition: The user does not know how to solve their issue. \\

{\bf Main Success Scenario} \\
Step 1: A user makes a post on the LSST Community web forum detailing issues they are having in querying the Object catalog from the latest data release (their Jupyter Notebook will not run.) \\
Step 2: Other users and an SP-CST member respond to the post with advice about their query. \\
Step 3: Discussion thread on the post realizes the user needs more computational resources. \\
Step 4: The user applies to the Resource Allocation Committee for additional processing. \\

{\bf Extensions} \\
Step 4 Alteration: Optional addition, the SP-CST add this user's particular case to documenation about large queries to avoid such issues in the future. \\

{\bf Sub-Variations} \\
\\

{\bf Related Information} (Optional) \\
Priority: Low  \\
Performance Target: a few days at most \\
Frequency: probably pretty often \\
Superordinate Use Case:  \\
Subordinate Use Cases: \\
Channel to primary actor: Interactive (web forum) \\
Secondary Actors: other community users \\
Channel to Secondary Actors: Interactive (web forum) \\

{\bf Open Issues} (Optional) \\

{\bf Schedule} \\
Due Date: \\
