\textbf{Use Case:} 004 Large Queries: Community Self-Help Using CST Resources \\

\textbf{Characteristic Information} \\
\textbf{Trigger:} A user's request for help posted on the LSST Community web forum. \\
\textbf{Goal in Context:} A user receives assistance in executing large queries on a Rubin data product. \\
\textbf{Primary Actor:} The System Performance Community Engagement Team (SP-CST) \\
\textbf{Scope:} SP-CST \\
\textbf{Level:} \\
\textbf{Preconditions:} In-operations, with users accessing the Rubin Science Platform Notebook Aspect, for which training tutorials exist and are served by the SP-CST. \\
\textbf{Success End Condition:} The user has a path towards solving their issue. \\
\textbf{Failed End Condition:} The user does not know how to solve their issue. \\

\textbf{Main Success Scenario} \\
Step 1: A user makes a post on the LSST Community web forum detailing issues they are having in querying the Object catalog from the latest data release (their Jupyter Notebook will not run). \\
Step 2: Other users and an SP-CST member respond to the post with advice about their query. \\
Step 3: Discussion thread on the post identifies that the user needs more computational resources. \\
Step 4: The user applies to the Resource Allocation Committee for additional processing resources. \\

\textbf{Extensions} \\
Step 4 Alteration: Optional addition, the SP-CST adds this user's particular case to documentation about large queries to help avoid similar issues in the future. \\

\textbf{Sub-Variations} \\
\\

\textbf{Related Information} (Optional) \\
\textbf{Priority:} Low \\
\textbf{Performance Target:} A few days at most \\
\textbf{Frequency:} Probably pretty often \\
\textbf{Superordinate Use Case:} \\
\textbf{Subordinate Use Cases:} \\
\textbf{Channel to Primary Actor:} Interactive (web forum) \\
\textbf{Secondary Actors:} Other community users \\
\textbf{Channel to Secondary Actors:} Interactive (web forum) \\

\textbf{Open Issues} (Optional) \\

\textbf{Schedule} \\
\textbf{Due Date:} \\
